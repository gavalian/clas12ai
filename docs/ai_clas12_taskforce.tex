%\documentclass[12pt]{article}
\documentclass[11pt]{revtex4}
\usepackage[dvipdf]{graphicx}
\usepackage{color}
\usepackage{multirow}
\usepackage{subfigure}  % For subfloats
\usepackage{epsfig}
\usepackage{wrapfig}
\usepackage{rotating}
\usepackage{graphicx}
\usepackage{amsmath}
%
\newcommand{\JLAB}{Thomas Jefferson National Accelerator Facility, Newport News, Virginia 23606}
\newcommand{\JP}{J/$\psi~$}
%
\begin{document}
\title{ Artificial Intelegence Task Force Report}


\author{G. Gavalian}
\affiliation{Thomas Jefferson National Accelerator Facility, Newport News, VA 23606}
%

\date{\today}

\begin{abstract}
%{\footnotesize {\bf Abstract}   }
{{\bf Abstract.}
This document describes AI projects that we think can help CLAS12 workflow.
}
\end{abstract}
\maketitle
%\end{document}
%\tableofcontents
%\clearpage

\vspace{1in}

\section{Introduction}

In this document we provide possible areas where AI can aid CLAS12 software
to improve either performance or reliability.

\section{Tracking}

During past year AI was introduced into tracking to help identify correct
track candidates based on the segments in drift chambers. The project was
great success in identifying correct segment combinations with reliability
of above 99\%, and lead to reconstruction speed reduction of factor of 6.

The Recurrent Neural Networks (RNN) were also implemented which gives a good prediction
of segments in the first 2 super layers based on segments in 4 other super layers.
This is done by using Long Short Term Memory (LSTM) network for series predictions.
After the first successful step we would like to expand this network architecture
to predict state vector for the track in order to reduce number of iterations
of Kalman-Filter necessary to converge on the track. This can lead to further
reduction in tracking time.

\section{Electron Identification (Triggering)}

Neural Networks can also be used in identifying events where electron is detected.
Using prediction of series extended to the surface of Electromagnetic Calorimeter
(ECAL), we can identify events where track has a matching hit in ECAL and has
enough energy to be identified as an electron. Using this prediction we can significantly
reduce number of events that should be analyzed (by tracking algorithms). This
will lead to further reduction of tracking time.

\section{Online Detector Health Monitoring}

Online monitoring of detector components can also be substituted by AI. It was
already successfully deployed in small scale in experimental Hall D. We can
develop a Neural Network that can be trained on the good runs to recognize
detector occupancy patterns and run classifier network to identify events
(collection of events accumulated over some time) as good or bad.

\section{Online Data Calibration}

We haven't investigated possibility of calibrating the detector with Neural
Networks. But it is a possibility and we have to look into it.

\section{Summary}

In summary several areas are identified where AI can be helpful for either
data processing or online data monitoring. These include:

\begin{itemize}
\item Charged particle track identification from track segments in Drift Chambers.
\item Determine track parameters from track segments, such as momentum and angles.
\item Online detector monitoring, which can extend to run validation in the offline as well.
\item Data filtering (level 3 trigger), to select events with possible trigger particle.
\item Online detector calibration, automatically adjust gains and calibration constant for detector.
\end{itemize}

Some of these projects have started and progressed, but they are in the development stage and
will require some further work to finalize the procedure and implement it in the workflow.
Other projects mentioned are speculations at this point and require some investigative work
to assess their feasibility.

\subsection{Investment and Time}

The AI tracking project for track classification and identification has started in collaboration
with ODU CRTC group, and progressed well during past year. It will be beneficial to continue
our collaboration with ODU, to use their students (who are very efficient in AI) and
work on developing the techniques that can work for our purposes.

To finalize first two projects, i.e. particle classification and prediction with RNN,
we would like to have students working on it for another year (2020-2021), during
that time we can also investigate other applications of AI mentioned in the list.

\end{document}
